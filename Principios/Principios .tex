\documentclass[]{article}
\usepackage[utf8]{inputenc}  % Soporte para caracteres UTF-8
\usepackage[spanish]{babel}  % Soporte para el idioma español
\usepackage{amsmath}
\usepackage{graphicx}
\usepackage{caption}
\usepackage{geometry}
\usepackage{hyperref}

\geometry{a4paper, margin=1in}

\title{Electrónica de Papel}
\author{Sebastián Espinoza Domínguez}
\date{\today}

\begin{document}

\maketitle

\section{Introducción}
En este documento abordaré la creación de dispositivos avanzados hechos completamente a partir de papel y grafito de lápiz,
 lo que permite la capacidad de dibujar circuitos y algoritmos.

\section{Componentes}
Lista de componentes hechos a partir de papel y grafito en 2D.

\subsection{Transistor}
Los transistores, en su forma más simplificada, parecen una "Y" mayúscula con dos líneas horizontales.
La conexión no varía mucho respecto a un transistor convencional.

\subsection{Resistencia}
Son muy fáciles de crear, basta con aumentar el grosor de la línea o disminuir la cantidad de grafito. A menor deposición
de grafito, menor será la conductividad.

\subsection{Conexión}
Las conexiones son simplemente líneas marcadas con grafito que permiten la conductividad.

\subsection{Diodo}
Los diodos tienen forma de "H", con la línea horizontal un poco más arriba.

\subsection{Bobinas}
Tienen forma de espirales, es decir, enrolladas sobre sí mismas. Están compuestas de dos tipos:

\begin{itemize}
    \item \textbf{Altavoz}: Parecen bastones enrollados sobre sí mismos, asemejándose a las plantas de la película de Jack
    de Tim Burton.
    \item \textbf{Inductor}: Tienen una forma más cuadrada, con ángulos exactos de 90º. Utilizan la distancia entre ellos 
    como condensador.
\end{itemize}

\subsection{Condensador}
Hay dos formas de hacerlo, sin embargo, ninguna me ha convencido y aún no determino las reglas correctas para su manejo.

\begin{itemize}
    \item \textbf{2D}: Se dibujan dos líneas paralelas entre sí, a una distancia muy pequeña, utilizando el papel como
    almacén de energía.
    \item \textbf{Doblez}: Se crea un doblez similar al origami. Se hace una marca en el área que será la parte inferior
    y otra en la superior, dejando un espacio en la parte inferior para el paso de la terminal superior, que se doblará 
    sobre la primera. Al añadir un poco de cinta adhesiva o material similar en medio, funciona como un condensador convencional.
\end{itemize}

\subsection{Compuertas Lógicas}
    Se pueden crear fácilmente a partir de transistores. Sin embargo, puede ser complicado conectarlas manualmente debido al cuidado
    con las entradas y salidas. Para facilitar esto, he estado utilizando software de diseño de circuitos con herramientas de
    autoconexión, donde se definen las conexiones, y como estas son básicamente las mismas, es fácilmente adaptable.

\begin{itemize}
    \item \textbf{Memoria}: Claramente, la memoria de papel no es fiable; es muy sensible y ocupa mucho más espacio
     en comparación con las memorias convencionales. Sin embargo, para programas sencillos, puede ser útil. Recomiendo
    evitar el uso de condensadores, ya que a mayor complejidad, menor fiabilidad. Es mejor usar arreglos NAND o flip-flops SR.
    \item \textbf{Otros circuitos}: Se pueden hacer desde calculadoras, multiplexores, etc. Es un proceso laborioso
    y al principio puede ser un tanto lento.
\end{itemize}

\section{Actuadores de Papel}
Los actuadores de papel son útiles para la robótica a un costo extremadamente bajo, aunque no son muy efectivos
por sí solos debido a su naturaleza. Sirven principalmente para presentaciones, pero yo los usaría en combinación
 con otros elementos, como microfluidos o autoensamble.

Con "actuadores" me refiero a que se mueven; sin embargo, en comparación con los componentes anteriores, esto requiere
 mucha más energía, normalmente alcanzada con 30V y repulsión magnética.

A continuación, describo distintos patrones y sus efectos:


    \subsection {Lineal} Se dibuja una línea sobre sí misma. Debería parecerse a una "n" en las partes negativa y positiva.
     Cuando fluye suficiente corriente, se doblará en la dirección de la marca.
    \subsection {Zigzag} Provoca contracciones, aunque no son muy impresionantes. Si aumentas la corriente tras observar el
     primer efecto, es probable que se queme.
    \subsection {Enrollar} Este patrón sirve para sujetar objetos. Simplemente haz lo mismo que en el patrón lineal, pero
     en una tira más larga y enróllela. Se enrollará bien, pero no ejercerá mucha presión.
    \subsection {Cuadrícula}Funciona principalmente como sensor para detectar toques.

Nota: Todos estos patrones funcionan como sensores táctiles.

Las bobinas, al ser fabricadas, pueden generar vibraciones similares a las de un altavoz, por lo que son muy rápidas.
 No he comprobado esto, pero he podido hacer pequeños robots voladores con componentes normales.

\section{Recomendaciones}
Pueden usarlos como deseen; se pueden crear cosas entretenidas al combinarlos adecuadamente. Sin embargo, recomiendo 
aplicar estos principios en detalles pequeños y estéticos, como mover pequeñas decoraciones o ensamblar cosas.

\section{A Futuro}
Actualmente, estoy trabajando en un procesador muy sencillo que sea programable. La meta es crear uno de 8 bits con al menos 5 
kBytes de memoria programable mediante un multiplexor. Además, quiero diseñarlo en forma de bloques para que el usuario final
pueda agregar o quitar entradas y salidas. También planeo desarrollar una herramienta para compilar un diseño escrito en un
prototipo viable, similar a los lenguajes de descripción de hardware en FPGA.

\end{document}
