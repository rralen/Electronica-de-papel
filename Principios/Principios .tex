\documentclass[]{article}
\usepackage[utf8]{inputenc}  % Soporte para caracteres UTF-8
\usepackage[spanish]{babel}  % Soporte para el idioma español
\usepackage{amsmath}
\usepackage{graphicx}
\usepackage{caption}
\usepackage{geometry}
\usepackage{hyperref}


\geometry{a4paper, margin=1in}

\title{Electronica de papel}
\author{Sebastián Espinoza Domínguez}
\date{\today}

\begin{document}

\maketitle

\section{Introduccion}
bueno en este documento abordare la creacion de dispositivos 
abanzados echos completamente apartir de papel y grafito
lapiz dando la capacidad de dibujar algoritmos

\section{Componentes}
lista de componentes echos apartir de papel y grafito en 2d
\subsection{Transistor}
los transistores en su forma mas simplificada parece una "Y"
mayuscula con dos lineas horizontales,La conexion no varia mucho
a un transistor convencional
\subsection{Recistensia}
Son muy faciles de crear bastando con aumentar el grosor de la linea
o disminuyendo la cantidad de grafito debido a que a 
menor deposicion de grafito menor conductividad
\subsection{Conexion}
Son solo lineas remarcadas de grafito permitiendo la conductividad
\subsection{Diodos}Los diosdos tienen forma de "H" con la linea 
horizontal un poco mas arriba
\subsection{Bobinas}Formas de espirales es decir enrolladas sobre
si mismas estan compuestos de dos:
\begin{itemize}
    \item \textbf{Altavoz}Parecen bastones enrollados sobre si
    asemejandose a las plantas de la pelicula de jack de 
    tim burton
    \item \textbf{Inductor}tienen una forma mas cuadrada con 
    angulos exactos de 90º y usan la distancia entre si como
    Condensador
\end{itemize}
\subsection{Condensador}
hay dos dfromas de acerlo sin embargo ninguna me a convencido
y aun no determino las reglas correctas para manejarlas
\begin{itemize}
    \item \textbf{2D}
    Se dibujan dos lineas paralelas entre si a una distancia muy
    pequeña entre si usando el papel como almacenaje de energuia
    \item \textbf{Doblez}
    Crear un doblez que se asemeje a origami: se hace una marca en el área que será la parte
    inferior y otra en la superior, dejando un espacio en la parte inferior para el paso de la
    terminal superior, que se doblará sobre la primera. Al añadir un poco de cinta adhesiva o
    material similar en medio, funciona como un condensador convencional.
\end{itemize}
\subsection{Compuertas Logicas}
se Pueden
\begin{itemize}
    \item \textbf{Memoria}
    \item \textbf{Otros circuitos}
\end{itemize}

\section{Actuadores}




\end{document}